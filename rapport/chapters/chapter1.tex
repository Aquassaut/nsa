\chapter{Données}

\paragraph{}
    Pour notre programme, nous avons décidé de permettre à l'utilisateur
    d'entrer des données de trois façon
    \begin{itemize}
        \item Avec un fichier
        \item Avec une génération aléatoire de données
        \item Interactivement
    \end{itemize}

    \subsection{Fichier}
        Le fichier, pour être interprété correctement par notre programme
        doit être au format JSON et comprendre :
        \begin{itemize}
            \item Une liste de zones, nommée "zones"
            \item Une liste de capteurs nommée "capteurs"
        \end{itemize}
        De plus chaque capteur doit comporter les champs suivants :
        \begin{itemize}
            \item Une liste de zones couvertes, nommée "zones"
            \item Une durée de vie, nommée "lifetime"
        \end{itemize}

    \subsection{Génération aléatoire}
        L'utilisateur peut demander à procéder à une génération aléatoire des
        données soit en fournissant une graine qui permettra d'obtenir plusieurs
        fois le même résultat, soit en laissant l'application choisir une graine
        pour lui, auquel cas l'application affichera la graine pour que
        l'utilisateur puisse la réutiliser.


    \subsection{Interactivement}
        Dans ce mode de création de données, l'utilisateur rentre les
        informations relatives aux zones, puis à chaque capteur, de façon
        manuelle. Ce mode de création est déconseillé car il est relativement
        chronophage.
