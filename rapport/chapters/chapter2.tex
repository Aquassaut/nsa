\chapter{Configurations}

\paragraph{}
    Afin de trouver des configurations élémentaires à analyser, nous avons mis
    au point plusieurs méthodes de calcul permettant de trouver un équilibre
    entre la précision souhaitée et le temps d'execution.
    Les méthodes que nous avons développées sont les suivantes.
    \begin{itemize}
        \item L'algorithme général
        \item La méthode des paires
        \item La méthode accumulative
        \item La méthode complétive
    \end{itemize}

\subsection{L'algorithme général}
    \paragraph{}
    Cet algorithme permet de trouver toutes les configurations pour un jeu de
    données.
    Il consiste à créer autant d'ensembles qu'il y a de zones à observer.
    Ceci fait, on attribue ajoute chaque capteur dans le ou les seau(x)
    correspondant à la ou aux zones que ce capteur observe.
    Enfin, nous calculons le produit cartésien de ces ensemble pour trouver la
    solution finale.
    \begin{description}
        \item[Forces de l'algorithme]\hfill \\
            Cet algorithme permet de trouver la liste exhaustive des solutions
            possibles.
        \item[Défauts de l'algorithme]\hfill \\
            Cet algorithme est très peu efficace du point de vue du temps
            d'execution, le produit cartésien ayant une complexité relativement
            élevée.
    \end{description}

\subsection{La méthode des paires}
    \paragraph{}
        Cet algorithme consiste à chercher toutes les paires de capteurs formant
        des configurations valides.
        Pour ce faire, nous considérons un capteurs, et cherchons parmi les
        autres capteurs tous les capteur pouvant former une configuration en
        conjonction.
        Nous continuons à proceder de la sorte jusqu'à avoir obtenu toutes les
        paires
    \begin{description}
        \item[Forces de l'algorithme]\hfill \\
            Cet algorithme permet de trouver la liste exhaustive des solutions
            possibles dans son domaine d'application, il s'agit donc d'un bon
            algorithme si les contraintes imposent de minimiser le nombre de
            capteurs allumés simultanément.
        \item[Défauts de l'algorithme]\hfill \\
            Cet algorithme ne trouvera pas ou peu de configuration valable dans
            les cas où l'on a beaucoup de zone et que la portée des capteurs est
            limitée.
    \end{description}

\subsection{La méthode accumulative}
    \paragraph{}
        Cet algorithme vise à obtenir une complexité très faible (O(n)) en ne
        parcourant qu'une fois la liste des capteurs.
        Elle consiste à parcourir la liste de capteurs en les ajoutant l'un
        après l'autre à la configuration courante, jusqu'à ce que l'ensemble des
        capteurs soient couvert. On passe alors à la configuration suivante et
        on recommence à partir du capteur suivant.

    \begin{description}
        \item[Forces de l'algorithme]\hfill \\
            Cet algorithme a une très faible complexité, et permettra toujours
            de trouver au moins une configuration.
        \item[Défauts de l'algorithme]\hfill \\
            Le nombre de configurations trouvables avec cet algorithme est
            relativement faible.
    \end{description}

\subsection{La méthode complétive}
    \paragraph{}
        Cet algorithme consiste à trouver les capteurs observant le plus de
        zones et de les compléter par des capteurs de moins grande portée pour
        obtenir des configurations.
        Ainsi, nous classons les capteurs en deux groupes, en fonction du nombre
        de zones couvertes par ceux ci, et pour chaque capteur à grande portée,
        nous lui ajoutons autant de capteur de la liste des capteurs à petite
        portée.

    \begin{description}
        \item[Forces de l'algorithme]\hfill \\
            Cette méthode permet de trouver beaucoup de configurations si le
            nombre de zones couvertes par les capteurs connait de grande
            variation mais que les zones sont réparties de façon homogène entre
            les capteurs.
        \item[Défauts de l'algorithme]\hfill \\
            L'algorithme ne permet pas toujours de trouver de configurations,
            en particulier lorsque l'on a des capteurs à grande portée sur
            l'ensemble des zones mais que les capteurs à petite portée ne sont
            présente que sur un petit sous-ensemble des zones traitées.
    \end{description}
